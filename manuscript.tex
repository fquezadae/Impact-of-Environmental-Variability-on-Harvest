% Options for packages loaded elsewhere
\PassOptionsToPackage{unicode}{hyperref}
\PassOptionsToPackage{hyphens}{url}
\PassOptionsToPackage{dvipsnames,svgnames,x11names}{xcolor}
%
\documentclass[
  11pt,
]{article}
\usepackage{amsmath,amssymb}
\usepackage{iftex}
\ifPDFTeX
  \usepackage[T1]{fontenc}
  \usepackage[utf8]{inputenc}
  \usepackage{textcomp} % provide euro and other symbols
\else % if luatex or xetex
  \usepackage{unicode-math} % this also loads fontspec
  \defaultfontfeatures{Scale=MatchLowercase}
  \defaultfontfeatures[\rmfamily]{Ligatures=TeX,Scale=1}
\fi
\usepackage{lmodern}
\ifPDFTeX\else
  % xetex/luatex font selection
\fi
% Use upquote if available, for straight quotes in verbatim environments
\IfFileExists{upquote.sty}{\usepackage{upquote}}{}
\IfFileExists{microtype.sty}{% use microtype if available
  \usepackage[]{microtype}
  \UseMicrotypeSet[protrusion]{basicmath} % disable protrusion for tt fonts
}{}
\makeatletter
\@ifundefined{KOMAClassName}{% if non-KOMA class
  \IfFileExists{parskip.sty}{%
    \usepackage{parskip}
  }{% else
    \setlength{\parindent}{0pt}
    \setlength{\parskip}{6pt plus 2pt minus 1pt}}
}{% if KOMA class
  \KOMAoptions{parskip=half}}
\makeatother
\usepackage{xcolor}
\usepackage[margin=1in]{geometry}
\usepackage{graphicx}
\makeatletter
\newsavebox\pandoc@box
\newcommand*\pandocbounded[1]{% scales image to fit in text height/width
  \sbox\pandoc@box{#1}%
  \Gscale@div\@tempa{\textheight}{\dimexpr\ht\pandoc@box+\dp\pandoc@box\relax}%
  \Gscale@div\@tempb{\linewidth}{\wd\pandoc@box}%
  \ifdim\@tempb\p@<\@tempa\p@\let\@tempa\@tempb\fi% select the smaller of both
  \ifdim\@tempa\p@<\p@\scalebox{\@tempa}{\usebox\pandoc@box}%
  \else\usebox{\pandoc@box}%
  \fi%
}
% Set default figure placement to htbp
\def\fps@figure{htbp}
\makeatother
\setlength{\emergencystretch}{3em} % prevent overfull lines
\providecommand{\tightlist}{%
  \setlength{\itemsep}{0pt}\setlength{\parskip}{0pt}}
\setcounter{secnumdepth}{5}
% definitions for citeproc citations
\NewDocumentCommand\citeproctext{}{}
\NewDocumentCommand\citeproc{mm}{%
  \begingroup\def\citeproctext{#2}\cite{#1}\endgroup}
\makeatletter
 % allow citations to break across lines
 \let\@cite@ofmt\@firstofone
 % avoid brackets around text for \cite:
 \def\@biblabel#1{}
 \def\@cite#1#2{{#1\if@tempswa , #2\fi}}
\makeatother
\newlength{\cslhangindent}
\setlength{\cslhangindent}{1.5em}
\newlength{\csllabelwidth}
\setlength{\csllabelwidth}{3em}
\newenvironment{CSLReferences}[2] % #1 hanging-indent, #2 entry-spacing
 {\begin{list}{}{%
  \setlength{\itemindent}{0pt}
  \setlength{\leftmargin}{0pt}
  \setlength{\parsep}{0pt}
  % turn on hanging indent if param 1 is 1
  \ifodd #1
   \setlength{\leftmargin}{\cslhangindent}
   \setlength{\itemindent}{-1\cslhangindent}
  \fi
  % set entry spacing
  \setlength{\itemsep}{#2\baselineskip}}}
 {\end{list}}
\usepackage{calc}
\newcommand{\CSLBlock}[1]{\hfill\break\parbox[t]{\linewidth}{\strut\ignorespaces#1\strut}}
\newcommand{\CSLLeftMargin}[1]{\parbox[t]{\csllabelwidth}{\strut#1\strut}}
\newcommand{\CSLRightInline}[1]{\parbox[t]{\linewidth - \csllabelwidth}{\strut#1\strut}}
\newcommand{\CSLIndent}[1]{\hspace{\cslhangindent}#1}
\usepackage{setspace}
\onehalfspacing
\usepackage{indentfirst}
\setlength{\parindent}{10pt}
\usepackage{authblk}
\author{Felipe J. Quezada}
\affil{Department of Economics \\ University of Concepción \vspace{-48pt}}
\usepackage{bookmark}
\IfFileExists{xurl.sty}{\usepackage{xurl}}{} % add URL line breaks if available
\urlstyle{same}
\hypersetup{
  pdftitle={The Impact of Environmental Variability on Fishers' Harvest Decisions in Chile using a Multi-Species Approach},
  pdfauthor={Felipe J. Quezada},
  colorlinks=true,
  linkcolor={blue},
  filecolor={Maroon},
  citecolor={blue},
  urlcolor={blue},
  pdfcreator={LaTeX via pandoc}}

\title{The Impact of Environmental Variability on Fishers' Harvest
Decisions in Chile using a Multi-Species Approach}
\author{Felipe J. Quezada}
\date{May 07, 2025}

\begin{document}
\maketitle
\begin{abstract}
In this paper, we aim to answer how fishing decisions, aggregate catch
levels, and the price of marine resources will be affected under
different climatic scenarios in the multi-species small pelagic fishery
(SPF) in Chile, composed by anchoveta (Engraulis ringens), jack mackerel
(Trachurus murphyi), sardine (either Sardinops sagax or Strangomera
bentincki), among others. By doing this, we expect to understand better
how Chilean fishers and fishing communities will adapt to climate
change. To address our research question, we will estimate a
multi-species harvesting model. This model considers species' economic
and biological interrelation to study the effect of climate variability
on harvest decisions and substitution between species and determine the
impact of different climatic scenarios on the well-being (e.g., profits)
of fishers and fishing communities in Chile. We hypothesize that if the
availability of a main target species is reduced, fishers will switch to
the closest substitute if the expected revenue obtained from targeting
this new species is high enough to cover the expected cost. Otherwise,
the vessel would decrease fishing efforts or even exit the fishery due
to the lack of economically viable substitutes. Moreover, we expect that
this behavior is heterogeneous depending on the geographical area of
operation -- as it determines the availability of other species-- and
the gear type used.
\end{abstract}

\section{Introduction}\label{introduction}

The distribution and abundance of marine resources are changing in
response to environmental conditions such as global ocean warming
(\citeproc{ref-Poloczanska2013-qq}{Poloczanska et al., 2013}). Due to
climate change, species distribution is expected to change in the
future, reducing species availability in some areas but increasing in
others. The literature that studies fishermen' responses to either
changes in fish availability or policies that restrict access to
fisheries (e.g., \citeproc{ref-Stafford2018-pq}{Stafford, 2018};
\citeproc{ref-Vasquez_Caballero2023-ip}{Vasquez Caballero et al., 2023})
has identified that fishers can adopt the following adaptive strategies:
(i) fishermen can reduce or reallocate fishing effort, either to another
species or to another location
(\citeproc{ref-Gonzalez-Mon2021-kj}{Gonzalez-Mon et al., 2021}), (ii)
keep following the same strategy, or, (iii) in the worst-case scenario,
stop fishing entirely and find alternative employment elsewhere
(\citeproc{ref-Powell2022-wj}{Powell et al., 2022}). Among all those
strategies, reallocating fishing efforts to other alternative species
might be an effective adaptation strategy to climate change
(\citeproc{ref-Young2018-kk}{Young et al., 2018}). Diversification of
target species has been associated with reducing income variability
(e.g., \citeproc{ref-Kasperski2013-jz}{Kasperski \& Holland, 2013};
\citeproc{ref-Sethi2014-bn}{Sethi et al., 2014}) and increasing
resilience to both climate shock (\citeproc{ref-Cline2017-dp}{Cline et
al., 2017}; \citeproc{ref-Fisher2021-lw}{Fisher et al., 2021}) and
interannual oceanographic variability
(\citeproc{ref-Aguilera2015-wo}{Aguilera et al., 2015};
\citeproc{ref-Finkbeiner2015-bs}{Finkbeiner, 2015}).

However, switching between species requires fishers to have the skills,
the gear, and the permits to do so
(\citeproc{ref-Frawley2021-cw}{Frawley et al., 2021};
\citeproc{ref-Powell2022-wj}{Powell et al., 2022}). Moreover, even
though a fisher may satisfy these requirements, diversification might
not be possible (\citeproc{ref-Beaudreau2019-xg}{Beaudreau et al.,
2019}) as it might be constrained depending on port infrastructure,
markets, and regulations (\citeproc{ref-Kasperski2013-jz}{Kasperski \&
Holland, 2013}; \citeproc{ref-Powell2022-wj}{Powell et al., 2022}).
Therefore, deciding which adaptation strategy to take is not
straightforward and would depend on many factors. Additionally, fishers
might respond differently to an analogous situation as they have
different goals, skills, and preferences
(\citeproc{ref-Jardine2020-um}{Jardine et al., 2020};
\citeproc{ref-Powell2022-wj}{Powell et al., 2022};
\citeproc{ref-Zhang2011-wv}{Zhang \& Smith, 2011}).

In this project, we aim to answer how fishing decisions, aggregate catch
levels, and the price of marine resources will be affected under
different climatic scenarios in the multi-species small pelagic fishery
(SPF) in Chile, composed by anchoveta (\emph{Engraulis ringens}), jack
mackerel (\emph{Trachurus murphyi}), sardine (either \emph{Sardinops
sagax} or \emph{Strangomera bentincki}), among others. The SPF is the
most important in terms of catches in the country, accounting for almost
94\% of the total Chilean catch in 2019
(\citeproc{ref-SUBPESCA2020}{SUBPESCA, 2020}). By doing this research,
we expect to understand better how Chilean fishers and fishing
communities will adapt to climate change. To address our research
question, we will estimate a multi-species harvesting model based on
Kasperski (\citeproc{ref-Kasperski2015-jm}{2015}). This model considers
species' economic and biological interrelation to study the effect of
climate variability on harvest decisions and substitution between
species and determine the impact of different climatic scenarios on the
well-being (e.g., profits) of fishers and fishing communities in Chile.

We hypothesize that if the availability of a main target species is
reduced, fishers will switch to the closest substitute if the expected
revenue obtained from targeting this new species is high enough to cover
the expected cost. Otherwise, the vessel would decrease fishing efforts
or even exit the fishery due to the lack of economically viable
substitutes. Moreover, we expect that this behavior is heterogeneous
depending on the geographical area of operation -- as it determines the
availability of other species (\citeproc{ref-Reimer2017-jw}{Reimer et
al., 2017}) -- and the gear type used.

At the end of the project, I expect to find significant effects of
climate variables on species stock dynamics, the cost of fishing during
a trip, and the number of trips a vessel would take. The combinations of
these environmental effects would be reflected in the optimal harvest
level and the prices seen on the local market. I also expect to find
significant interrelations between species stock and harvest, and that
the composition of the catch will vary depending on the climate scenario
we use for future predictions.

Under a changing climate, studying the effect of climatic variability on
fishers' harvest decisions and landings is relevant for understanding
fishing communities' adaptive capacities and strategies in response to
climate change, thereby enabling the design of potential mitigation
measures in response to these changes by policymakers. Countries have
different institutions, cultures, and norms, so responses might differ
depending on where the study is conducted. For this reason, conducting
this research based on the Chilean fishing industry is necessary to
develop local policies that aim to reduce climate change impacts on
fisheries. While there is some literature on the effect of climate
change on Chilean fisheries, I am unaware of local-level studies that
consider a multiple-species framework and the interrelationship between
the local market and fishing decisions seen under a variable climate
context.\footnote{For the case of Chile, as far as I know, the only
  article that study fishers' behavior using discrete choice modelling
  is Peña-Torres et al. (\citeproc{ref-Peuxf1aElNiuxf1o}{2017}). This
  article study how El Niño Southern Oscillation (ENSO) affect fishermen
  location choices that participate in the Jack Mackerel fishery.}

\section{Data and methodology}\label{data-and-methodology}

To fulfill the project's objectives, and following Kasperski
(\citeproc{ref-Kasperski2015-jm}{2015}), the research entails five
different stages: (i) estimating the stock dynamics of each species
included in the model, (ii) estimating trip level cost functions, (iii)
estimating total annual trips, (iv) estimate the inverse demand model
for outputs (i.e., price responses to supply), and (v) conduct numerical
optimization to examine how harvest and profits levels evolve over time.
The numeral optimization uses estimated parameters from the previous
four stages to conduct the optimization procedure.

\subsection{Data}\label{data}

\begin{itemize}
\tightlist
\item
  \textbf{SOLICITADO A IFOP 2012-2024:}

  \begin{itemize}
  \tightlist
  \item
    Stock abundance and vessel landings (annual by
    port/county/region/country and species)
  \item
    Data at the trip level (\href{ifop.dataobservatory.net}{IFOP data
    observatory?}{]}).
  \item
    Ex-vessel prices (monthly or annual by port/county/region/country
    and species)
  \end{itemize}
\item
  \textbf{POR SOLICITAR:}

  \begin{itemize}
  \tightlist
  \item
    Environmental covariates:

    \begin{itemize}
    \tightlist
    \item
      Sea surface temperature
    \item
      Chlorophyll levels
    \item
      Wind intensity and wave conditions in each trip at the harvest
      location
    \item
      Bad weather days?
    \end{itemize}
  \item
    Other data?

    \begin{itemize}
    \tightlist
    \item
      Average wage pay to crew member per hour
    \item
      Diesel cost.
    \end{itemize}
  \end{itemize}
\end{itemize}

\subsection{Econometrics models}\label{econometrics-models}

\subsubsection{Stock dynamics}\label{stock-dynamics}

To estimate stock dynamics, I use annual data on stock abundance and
vessel landings. Following Kasperski
(\citeproc{ref-Kasperski2015-jm}{2015}), the growth of each species
follows a discrete logistic function: \begin{equation}
x_{i,y+1} + h_{iy} = \underbrace{(1 + r_i)x_{iy} + \eta_i x_{iy}^2}_{R_i(x_{iy})} + \underbrace{\sum_{j \neq i}^{n-1} a_{ij} x_{iy} x_{jy}}_{I_i(x_y)} \quad i=1,\ldots,n
\label{eq1}
\end{equation}

where \(x_{iy}\) is the fish stock by species \(i=1,\ldots,n\) in year
\(y\), \(n\) is the total number of species, \(h_{iy}\) is the annual
harvest of species \(i\) on year \(y\), \(r_i\) is the intrinsic growth
rate of the resource \(i\), \(\eta_i\) is a density-dependent factor
related to the carrying capacity, and \(\alpha_{ij}\) are the
interaction parameters between species. The system of \(n\) growth
equations can be estimated simultaneously using seemingly unrelated
regression (SUR) or other similar approaches.

Following Richter et al. (\citeproc{ref-Richter2018}{2018}), we can
augment \eqref{eq1} by including environmental covariates \(Env_{iy}\)
that affect the fish stock, such as sea surface temperature and
chlorophyll levels, and an error term \(\varepsilon_{iy}\) that captures
random recruitment: \begin{equation}
x_{i,y+1} + h_{iy} = \underbrace{(1 + r_i)x_{iy} + \eta_i x_{iy}^2}_{R_i(x_{iy})} + \underbrace{\sum_{j \neq i}^{n-1} a_{ij} x_{iy} x_{jy}}_{I_i(x_y)} + \rho_i Env_{iy} + \varepsilon_{iy} \quad i=1,\ldots,n
\label{eq2}
\end{equation} where \(\rho_i\) are the coefficient for the
environmental covariates. The model could also be expanded to different
spatial locations conditional on data availability.

\subsubsection{Trip level cost
functions}\label{trip-level-cost-functions}

Ignoring trip subscript, the cost functions vary by vessel
\(v=1,\ldots,V_g\) and gear used \(g=1,\ldots,G\), where \(V_g\) is the
number of observations using gear type \(g\), and \(G\) is the total
number of available (or observed) gears: \begin{equation}
C_{vg} = \sum_{i=1}^{2n+M+k} \alpha_{g, \mathbf{X}_i} \mathbf{X}_{ivg} + \frac{1}{2} \sum_{i=1}^{2n+M+k} \sum_{j=1}^{2n+M+k} \alpha_{g, \mathbf{X}_i\mathbf{X}_j} \mathbf{X}_{ivg} \mathbf{X}_{jvg}
\label{eq3}
\end{equation} where \(C_{vg}= w z_{vg}^*\),
\(\mathbf{X}_{vg}=[w;h_{vg};x;Z_v]\), \(w\) is a \(V_g \times M\) matrix
of variable input prices, \(h_{vg}\) is an \(V_g \times n\) matrix of
harvest quantities and \(Z_v\) is a vector of fixed vessel
characteristics for vessel v, x is an \(V_g \times n\) matrix of given
stock levels of the species of interest, and \(Z_v\) is an
\(V_g \times k\) matrix of given vessel characteristics. Therefore,
\(\mathbf{X}_{vg}\) is a \(V_g \times (2n+M+k)\) matrix, and
\(\mathbf{X}_{ivg}\) represents the \emph{i}th column of the
\(\mathbf{X}_{vg}\) matrix.

Together with estimating the restricted cost function, we estimate the
conditional input demand equations. This addition allows an increase in
the degrees of freedom by imposing cross-equation parameter constraints
and allows for the testing of, for instance, jointness in inputs
(\citeproc{ref-Kasperski2015-jm}{Kasperski, 2015}). The conditional
input demand equations are derived by Shepard's Lemma: \begin{equation}
\frac{\partial C_{vg}}{\partial w_m} = z^*_{vg,w_m} = \alpha_{g,w_m} + \sum_{j=1}^{2n+M+k} \alpha_{g,w_m,\mathbf{X}_j} \mathbf{X}_{jvg} \quad m=1,\ldots,M.
\label{eq4}
\end{equation}

Similar to stock dynamics, the system of equations formed by \eqref{eq3}
and \eqref{eq4} can be estimated using SUR. To comply with economic
theory, and to reduce even more the number of parameters to estimate,
the following restrictions are imposed when estimating \eqref{eq4}:

\begin{enumerate}
\def\labelenumi{\arabic{enumi}.}
\item
  Symmetry of the cost function, where \begin{equation*}
    \alpha_{g,\mathbf{X}_i\mathbf{X}_j} = \alpha_{g,\mathbf{X}_j\mathbf{X}_i} \quad \forall \quad i=1,\ldots,(2n + M + k); \ i \neq j; \ g = 1,\ldots, G.
    \end{equation*}
\item
  Linear homogeneity in input prices, where \begin{equation*}
    \sum_m^M \alpha_{g,w_m} = 1 \ \text{and} \ \sum_m^M \alpha_{g,w_m,\mathbf{X}_i} = 0 \quad i=1,\ldots,(2n + M + k); \ g = 1,\ldots, G.
    \end{equation*}
\end{enumerate}

Data at the trip level is available upon request from the Chilean
Fisheries Research Institute (IFOP), which registers geo-referenced
catch information on the Chilean fleet's fishing operation per trip (see
e.g. \citeproc{ref-Peuxf1aElNiuxf1o}{Peña-Torres et al., 2017} and
\href{ifop.dataobservatory.net}{IFOP data observatory}). As inputs we
can use the time spent at sea during a trip, where the price is the
average wage pay to crew member per hour, and the distance traveled
during a trip, where the price of distance traveled is the diesel cost.

To link this function to climate change, we can also include additional
environmental variables \(Env\) to \(\mathbf{X}_{vg}\) such as wind
intensity and wave conditions in each trip at the harvest location, upon
data availability. Therefore, the augmented \(X_{vg}\) matrix becomes
\(\mathbf{X}^{'}_{vg} = [w;h_{vg};x;Z_v;Env]\).

\subsubsection{Total annual trips}\label{total-annual-trips}

The number of trips a vessel will take in a given year for each gear
type used is assumed to follow a Poison distribution
(\citeproc{ref-Kasperski2015-jm}{Kasperski, 2015}): \begin{equation}
Pr\left[T^{*}_{vgy} = t_v\right] = \frac{exp^{-exp(U^{'}_{vg}\beta_g)}exp(U^{'}_{vg}\beta_g)^{t_v}}{t_v !}
\label{eq5}
\end{equation} where \(U_{vg}=[p,w,h_{vg},\bar{q},Z_{vg}]\) is a
\((3n+M+k+1)×V_g\) matrix of explanatory variables, \(\beta_g\) is a
\((3n+M+k+1)\times1\) matrix of coefficients to be estimated, \(t_v\) is
the number of trips taken by vessel \(v\) using gear type \(g\) in year
\(y\), and \(\bar{q}\) is the annual quota level. Additionally, we can
add the accumulation of ``bad weather days'' as an explanatory variable
to incorporate weather conditions into this decision, thus
\(U_{vg}=[p,w,h_{vg},\bar{q},Z_{vg}, Env]\)

\subsubsection{Inverse demand model for
outputs}\label{inverse-demand-model-for-outputs}

The price of each species is modeled using an inverse demand model,
which assumes weak separability between the species into consideration
and other products (\citeproc{ref-Kasperski2015-jm}{Kasperski, 2015}).
The price of a species \(i\) in year \(y\) is the following:

\begin{equation}
p_{iy} = \sum_j^n \gamma_j p_{j,y-1} + \gamma_{h_i} h_{iy} + \epsilon_{iy}, \quad i = 1,\ldots,n, \ j = 1,\ldots,n.
\label{eq6}
\end{equation}

The system formed by \eqref{eq6} can be estimated using maximum
likelihood. Note that harvest may be endogenous in this system due to
simultaneity. Kasperski (\citeproc{ref-Kasperski2015-jm}{2015}) solves
this by assuming that the TAC is exogenous, and the catch, in general,
is determined by this quota. We can relax this assumption by considering
that all variables in the inverse demand equations are endogenous by
estimating a vector autoregressive (VAR) model
(\citeproc{ref-juselius2006cointegrated}{Juselius, 2006}). In other
words, harvest \(h_{vg}\) has its own equations in the system.

\subsection{Numerical optimization}\label{numerical-optimization}

To obtain the effect of future climate variability on stock, harvest,
quota and profits, we conduct numerical optimization for different
climate scenarios using the parameters estimates for the stock dynamic,
cost functions, total annual trips and inverse demand equations. In each
year, a vessel maximizes profits by choosing their optimal number of
trips \(T_g\) and harvest levels per trip \(h_{g\tau}\) given a gear
type: \begin{align}
\max_{h_{gt}, T_g} \quad \pi_{vgt} & = \sum_{\tau=t}^{T_g} \rho^\tau \left\{ P(h) h_{g\tau} - C_g(h_{g\tau} | w, x, Z, Env) \right\} \quad \tau = t,\ldots, T_g \nonumber \\
\textbf{s.t} \quad q_{g,t+1} & = \omega \ast \bar{q} - \sum_{t=1}^{t} h_{gt} \geq 0, \quad t = 1, \dots, T-1, \quad g = 1, \dots, G
\label{eq7}
\end{align} where \(\rho\) is the intra-annual discount factor,
\(\omega\) is a vector of shares of \(\bar{q}\), and \(h_{lt}=0\) for
all \(l\neq g\). The vector of shares is obtained from historical data
on harvest. The optimal profit from the maximization problem in
\eqref{eq7} is denoted as \(\pi_{vgy}^* (p,w,x,Z,\bar{q},\omega, Env)\),
and \(h_{vgty}^*\) and \(T_{vgy}^*\) are the optimal choices harvest per
trip and total number of trips in year y for vessel v. To obtain the
optimal quota level, we must solve the social-planner optimization
problem to maximize the net value of the fishery by choosing the quota
levels per year and by species.

Following Kasperski (\citeproc{ref-Kasperski2015-jm}{2015}), the
optimization problem will be conducted for the next 25 years. I will use
different climate scenarios and compare different optimal outcomes
between them by using future projections for the environmental variables
included in the model.

\section{Results}\label{results}

NO RESULTS YET

\section{Discussion}\label{discussion}

\subsection{Potential extension of the
project}\label{potential-extension-of-the-project}

Several other extensions to the model can be incorporated to be
improved. For instance, the geographical space where fishermen operate
is relevant, as depending on the location chosen and when to
participate, the set of potential choices would vary
(\citeproc{ref-Reimer2017-jw}{Reimer et al., 2017}). As I mentioned
above, it is possible to extend the stock dynamic model by considering
different locations. The model would also require that the participation
decision, which is captured by the Poisson model on the annual number of
trips, should then consider the decision to participate in a determined
fishing ground, connecting the multi-species model of Kasperski
(\citeproc{ref-Kasperski2015-jm}{2015}) to the literature of location
choice modeling (e.g., \citeproc{ref-Dupont1993-jn}{Dupont, 1993};
\citeproc{ref-Hicks2020-mz}{Hicks et al., 2020};
\citeproc{ref-Smith2005-us}{Smith, 2005}).

\section{Conclusions}\label{conclusions}

NO CONCLUSION YET

\section{Repository}\label{repository}

The source code for this project is available on
\href{https://github.com/fquezadae/Impact-of-Environmental-Variability-on-Harvest}{GitHub}

\section*{References}\label{references}
\addcontentsline{toc}{section}{References}

\phantomsection\label{refs}
\begin{CSLReferences}{1}{0}
\bibitem[\citeproctext]{ref-Aguilera2015-wo}
Aguilera, S. E., Cole, J., Finkbeiner, E. M., Le Cornu, E., Ban, N. C.,
Carr, M. H., Cinner, J. E., Crowder, L. B., Gelcich, S., Hicks, C. C.,
Kittinger, J. N., Martone, R., Malone, D., Pomeroy, C., Starr, R. M.,
Seram, S., Zuercher, R., \& Broad, K. (2015). Managing small-scale
commercial fisheries for adaptive capacity: Insights from dynamic
social-ecological drivers of change in monterey bay. \emph{PLoS One},
\emph{10}(3), e0118992.

\bibitem[\citeproctext]{ref-Beaudreau2019-xg}
Beaudreau, A. H., Ward, E. J., Brenner, R. E., Shelton, A. O., Watson,
J. T., Womack, J. C., Anderson, S. C., Haynie, A. C., Marshall, K. N.,
\& Williams, B. C. (2019). Thirty years of change and the future of
alaskan fisheries: Shifts in fishing participation and diversification
in response to environmental, regulatory and economic pressures.
\emph{Fish Fish}, \emph{20}(faf.12364), 601--619.

\bibitem[\citeproctext]{ref-Cline2017-dp}
Cline, T. J., Schindler, D. E., \& Hilborn, R. (2017). Fisheries
portfolio diversification and turnover buffer alaskan fishing
communities from abrupt resource and market changes. \emph{Nat.
Commun.}, \emph{8}, 14042.

\bibitem[\citeproctext]{ref-Dupont1993-jn}
Dupont, D. P. (1993). Price uncertainty, expectations formation and
fishers' location choices. \emph{Mar. Resour. Econ.}, \emph{8}(3),
219--247.

\bibitem[\citeproctext]{ref-Finkbeiner2015-bs}
Finkbeiner, E. M. (2015). The role of diversification in dynamic
small-scale fisheries: Lessons from baja california sur, mexico.
\emph{Glob. Environ. Change}, \emph{32}, 139--152.

\bibitem[\citeproctext]{ref-Fisher2021-lw}
Fisher, M. C., Moore, S. K., Jardine, S. L., Watson, J. R., \& Samhouri,
J. F. (2021). Climate shock effects and mediation in fisheries.
\emph{Proc. Natl. Acad. Sci. U. S. A.}, \emph{118}(2).

\bibitem[\citeproctext]{ref-Frawley2021-cw}
Frawley, T. H., Muhling, B. A., Brodie, S., Fisher, M. C., Tommasi, D.,
Le Fol, G., Hazen, E. L., Stohs, S. S., Finkbeiner, E. M., \& Jacox, M.
G. (2021). Changes to the structure and function of an albacore fishery
reveal shifting social‐ecological realities for pacific northwest
fishermen. \emph{Fish Fish}, \emph{22}(2), 280--297.

\bibitem[\citeproctext]{ref-Gonzalez-Mon2021-kj}
Gonzalez-Mon, B., Bodin, Ö., Lindkvist, E., Frawley, T. H., Giron-Nava,
A., Basurto, X., Nenadovic, M., \& Schlüter, M. (2021). Spatial
diversification as a mechanism to adapt to environmental changes in
small-scale fisheries. \emph{Environ. Sci. Policy}, \emph{116},
246--257.

\bibitem[\citeproctext]{ref-Hicks2020-mz}
Hicks, R. L., Holland, D. S., Kuriyama, P. T., \& Schnier, K. E. (2020).
Choice sets for spatial discrete choice models in data rich
environments. \emph{Res. Energy Econ.}, \emph{60}, 101148.

\bibitem[\citeproctext]{ref-Jardine2020-um}
Jardine, S. L., Fisher, M. C., Moore, S. K., \& Samhouri, J. F. (2020).
Inequality in the economic impacts from climate shocks in fisheries: The
case of harmful algal blooms. \emph{Ecol. Econ.}, \emph{176}, 106691.

\bibitem[\citeproctext]{ref-juselius2006cointegrated}
Juselius, K. (2006). \emph{The cointegrated VAR model: Methodology and
applications}. Oxford university press.

\bibitem[\citeproctext]{ref-Kasperski2015-jm}
Kasperski, S. (2015). Optimal multi-species harvesting in ecologically
and economically interdependent fisheries. \emph{Environ. Resour.
Econ.}, \emph{61}(4), 517--557.

\bibitem[\citeproctext]{ref-Kasperski2013-jz}
Kasperski, S., \& Holland, D. S. (2013). Income diversification and risk
for fishermen. \emph{Proc. Natl. Acad. Sci. U. S. A.}, \emph{110}(6),
2076--2081.

\bibitem[\citeproctext]{ref-Peuxf1aElNiuxf1o}
Peña-Torres, J., Dresdner, J., \& Vasquez, F. (2017). {El Niño and
Fishing Location Decisions: The Chilean Straddling Jack Mackerel
Fishery}. \emph{Marine Resource Economics}, \emph{32}(3), 249--275.
\url{https://doi.org/10.1086/692073}

\bibitem[\citeproctext]{ref-Poloczanska2013-qq}
Poloczanska, E. S., Brown, C. J., Sydeman, W. J., Kiessling, W.,
Schoeman, D. S., Moore, P. J., Brander, K., Bruno, J. F., Buckley, L.
B., Burrows, M. T., Duarte, C. M., Halpern, B. S., Holding, J., Kappel,
C. V., O'Connor, M. I., Pandolfi, J. M., Parmesan, C., Schwing, F.,
Thompson, S. A., \& Richardson, A. J. (2013). Global imprint of climate
change on marine life. \emph{Nat. Clim. Chang.}, \emph{3}(10), 919--925.

\bibitem[\citeproctext]{ref-Powell2022-wj}
Powell, F., Levine, A., \& Ordonez-Gauger, L. (2022). Climate adaptation
in the market squid fishery: Fishermen responses to past variability
associated with el niño southern oscillation cycles inform our
understanding of adaptive capacity in the face of future climate change.
\emph{Clim. Change}, \emph{173}(1-2), 1.

\bibitem[\citeproctext]{ref-Reimer2017-jw}
Reimer, M. N., Abbott, J. K., \& Wilen, J. E. (2017). Fisheries
production: Management institutions, spatial choice, and the quest for
policy invariance. \emph{Mar. Resour. Econ.}, \emph{32}(2), 143--168.

\bibitem[\citeproctext]{ref-Richter2018}
Richter, A., Eikeset, A. M., Van Soest, D., Diekert, F. K., \& Stenseth,
N. C. (2018). Optimal management under institutional constraints:
Determining a total allowable catch for different northeast arctic cod
fishery fleet segments. \emph{Environmental and Resource Economics},
\emph{69}, 811--835. \url{https://doi.org/10.1007/s10640-016-0106-3}

\bibitem[\citeproctext]{ref-Sethi2014-bn}
Sethi, S. A., Reimer, M., \& Knapp, G. (2014). Alaskan fishing community
revenues and the stabilizing role of fishing portfolios. \emph{Mar.
Policy}, \emph{48}, 134--141.

\bibitem[\citeproctext]{ref-Smith2005-us}
Smith, M. D. (2005). State dependence and heterogeneity in fishing
location choice. \emph{J. Environ. Econ. Manage.}, \emph{50}(2),
319--340.

\bibitem[\citeproctext]{ref-Stafford2018-pq}
Stafford, T. M. (2018). Accounting for outside options in discrete
choice models: An application to commercial fishing effort. \emph{J.
Environ. Econ. Manage.}, \emph{88}, 159--179.

\bibitem[\citeproctext]{ref-SUBPESCA2020}
SUBPESCA. (2020). \emph{Informe sectorial de pesca y acuicultura 2019}.
Subsecretaría de Pesca y Acuicultura.
\url{https://www.subpesca.cl/portal/618/articles-106845_documento.pdf}

\bibitem[\citeproctext]{ref-Vasquez_Caballero2023-ip}
Vasquez Caballero, S., Sylvia, G., \& Holland, D. S. (2023). Fishery
participation and location choice model: The west coast salmon troll
commercial fishery. \emph{Can. J. Fish. Aquat. Sci.}

\bibitem[\citeproctext]{ref-Young2018-kk}
Young, T., Fuller, E. C., Provost, M. M., Coleman, K. E., St. Martin,
K., McCay, B. J., \& Pinsky, M. L. (2018). Adaptation strategies of
coastal fishing communities as species shift poleward. \emph{ICES
Journal of Marine Science}, \emph{76}(1), 93--103.

\bibitem[\citeproctext]{ref-Zhang2011-wv}
Zhang, J., \& Smith, M. D. (2011). Heterogeneous response to marine
reserve formation: {A} sorting model approach. \emph{Environ. Resour.
Econ.}, \emph{49}(3), 311--325.

\end{CSLReferences}

\end{document}
