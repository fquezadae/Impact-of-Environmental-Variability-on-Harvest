\hypertarget{introduction}{%
\section{Introduction}\label{introduction}}

The distribution and abundance of marine resources are changing in
response to environmental conditions such as global ocean warming
{[}@Poloczanska2013-qq{]}. Climate change will shift species
distribution in the future, leading to reduced species availability in
some areas and increased availability in others. The literature that
studies fishermen' responses to either changes in fish availability or
policies that restrict access to fisheries {[}e.g., @Stafford2018-pq;
@Vasquez\_Caballero2023-ip{]} has identified that fishers can adopt the
following adaptive strategies: (i) fishermen can reduce or reallocate
fishing effort, either to another species or to another location
{[}@Gonzalez-Mon2021-kj{]}, (ii) keep following the same strategy, or,
(iii) in the worst-case scenario, stop fishing entirely and find
alternative employment elsewhere {[}@Powell2022-wj{]}. Among all those
strategies, reallocating fishing efforts to other alternative species
might be an effective adaptation strategy to climate change
{[}@Young2018-kk{]}. Researchers have linked the diversification of
target species to a reduction in income variability {[}e.g.,
@Kasperski2013-jz; @Sethi2014-bn{]}. This diversification also increases
resilience to both climate shocks {[}@Cline2017-dp; @Fisher2021-lw{]}
and interannual oceanographic variability {[}@Aguilera2015-wo;
@Finkbeiner2015-bs{]}.

\begin{itemize}
\tightlist
\item
  Climate variability affects income risk\ldots{} From @sjcruz\_jmp:

  \begin{itemize}
  \tightlist
  \item
    \emph{``Climate variability has a substantial impact on key
    food-producing sectors, particularly agriculture and fisheries,
    where income is closely tied to environmental fluctuations (e.g.,
    temperature and rainfall) and market forces (e.g., input costs)
    {[}@Kasperski2013-jz; @Carter2018{]}.''}
  \item
    \emph{``With climate variability expected to reduce productivity in
    these sectors, the associated income risk is likely to increase
    {[}@Carter2018; @Free2019{]}.''}
  \item
    \emph{``Adaptive strategies like diversification, both within (e.g.
    switching crop or species) and across sectors, are widely advocated
    to mitigate risks associated with climate variability
    {[}@Abbott2023-sb{]}, but these strategies can be costly,
    particularly for rural, resource-dependent communities with limited
    capital and skills {[}@Cherdchuchai2006; @Ellis2000{]}.''}
  \item
    \emph{``The role of these switching costs, and how they might hinder
    or aid diversification within and across sectors remain largely
    unexplored in this context''}.
  \end{itemize}
\end{itemize}

However, switching between species requires fishers to have the skills,
the gear, and the permits to do so {[}@Frawley2021-cw;
@Powell2022-wj{]}. Moreover, even though a fisher may satisfy these
requirements, diversification might not be possible
{[}@Beaudreau2019-xg{]} as it might be constrained depending on port
infrastructure, markets, and regulations {[}@Kasperski2013-jz;
@Powell2022-wj{]}. Therefore, deciding which adaptation strategy to
adopt is not straightforward and depends on many factors. Additionally,
fishers might respond differently to an analogous situation as they have
different goals, skills, and preferences {[}@Zhang2011-wv;
@Jardine2020-um; @Powell2022-wj{]}.

In this project, we aim to answer how fishing decisions, aggregate catch
levels, and the price of marine resources will be affected under
different climatic scenarios in the multi-species small pelagic fishery
(SPF) in Chile, composed of anchoveta (Engraulis ringens), jack mackerel
(Trachurus murphyi), sardine (either Sardinops sagax or Strangomera
bentincki), among others. The SPF is the most important in terms of
catches in the country, accounting for almost 94\% of the total Chilean
catch in 2019 {[}@SUBPESCA2020{]}. Through this research, we aim to gain
a deeper understanding of how Chilean fishers and fishing communities
will adapt to climate change. To address our research question, we will
estimate a multi-species harvesting model based on @Kasperski2015-jm.
This model considers species' economic and biological interrelations to
study the effect of climate variability on harvest decisions and
substitution between species, and to determine the impact of different
climatic scenarios on the well-being (e.g., profits) of fishers and
fishing communities in Chile.

We hypothesize that if the availability of a main target species
decreases, fishers will switch to the closest substitute if the expected
revenue obtained from targeting this new species is high enough to cover
the expected cost. Otherwise, the vessel would decrease fishing efforts
or even exit the fishery due to the lack of economically viable
substitutes. Moreover, we expect that this behavior is heterogeneous
depending on the geographical area of operation -- as it determines the
availability of other species {[}@Reimer2017-jw{]} -- and the gear type
used.

At the end of the project, I expect to find significant effects of
climate variables on species stock dynamics, the cost of fishing during
a trip, and the number of trips a vessel would take. The environmental
effects might influence the optimal harvest level and the prices seen in
the local market. I also expect to find significant interrelations
between species stock and harvest, and that the composition of the catch
will vary depending on the climate scenario we use for future
predictions.

Under a changing climate, studying the effect of climatic variability on
fishers' harvest decisions and landings is relevant for understanding
fishing communities' adaptive capacities and strategies in response to
climate change, thereby enabling the design of potential mitigation
measures in response to these changes by policymakers. Countries have
different institutions, cultures, and norms, leading to differing
responses based on the study's location. For this reason, conducting
this research based on the Chilean fishing industry is necessary to
develop local policies that aim to reduce climate change impacts on
fisheries. While there is some literature on the effect of climate
change on Chilean fisheries, I am unaware of local-level studies that
consider a multiple-species framework and the interrelationship between
the local market and fishing decisions seen under a variable climate
context.\footnote{For the case of Chile, as far as I know, the only
  article that study fishers' behavior using discrete choice modelling
  is @Pena-Torres2017-gn. This article study how El Niño Southern
  Oscillation (ENSO) affect fishermen location choices that participate
  in the Jack Mackerel fishery.}

\hypertarget{spf-in-chile}{%
\section{SPF in Chile}\label{spf-in-chile}}

The small pelagic fishery (SPF) in Chile is primarily composed of
anchoveta (\emph{Engraulis ringens}) and sardine (\emph{Strangomera
bentincki}), with jack mackerel (\emph{Trachurus murphyi}) also playing
a significant role, particularly for small-scale fishers engaged in a
``race for fish'' dynamic. The Central-South region of Chile is
especially important for sardine harvests and will therefore be the
focus of this research, as it provides a relevant setting to study
potential species substitution within a multispecies management
framework. In 2019, the SPF represented nearly 94\% of total national
fish landings, highlighting its critical importance to the Chilean
fisheries sector {[}@SUBPESCA2020{]}. Historically, anchoveta in the
Central-South was considered collapsed until 2018, shifted to
overexploited status in 2019, and has since been fished within maximum
sustainable yield (MSY) limits. Sardine stocks have generally remained
within MSY levels, except in 2021 and 2023 when they were classified as
overexploited. Similarly, jack mackerel was overexploited until 2018 but
has since been harvested sustainably. Anchoveta and sardine are
regulated as a mixed-species fishery: they have separate quotas, but if
one species is unavailable, the quota for the other can be used as a
substitute. Additionally, some quota originally allocated to industrial
fleets is transferred to the artisanal sector, with these transactions
potentially traceable through SERNAPESCA data.
\textless\textless\textless\textless\textless\textless\textless{} HEAD
=======

PREGUNTA: ¿¿¿¿Jurel al parecer otra especie, predadora de sancho?????

Sobre cuotas transables:

Since 2013, the industrial sector have been managed using individual
transferable quotas (ITQs), called Transferable Fishing Licenses (LTP).
These permits are individual catch quotas with a duration of 20 years,
renewable, and are separate from vessel ownership. Two types of LTP were
created: Class A LTP, which continued from the LMCA and were allocated
based on historical criteria as a percentage of the industrial fraction;
and Class B LTP, which are awarded through the auction of up to 15\% of
the industrial fraction. This auctioned fraction was obtained by an
equivalent reduction in Class A LTP. In 2015, the first auctions of this
type were held in various fishery units.

```\{r harvest\_substitution, message=FALSE, warning=FALSE,
include=FALSE\}

\hypertarget{first-get-your-original-summarized-table}{%
\section{First, get your original summarized
table}\label{first-get-your-original-summarized-table}}

vessel\_species\_year \textless- logbooks \%\textgreater\% filter(year
\textgreater= 2019) \%\textgreater\% \#\#\# all species available after
2019? select(COD\_BARCO, year, NOMBRE\_ESPECIE, CAPTURA\_RETENIDA)
\%\textgreater\% group\_by(COD\_BARCO, year, NOMBRE\_ESPECIE)
\%\textgreater\% summarize(total\_catch = sum(CAPTURA\_RETENIDA, na.rm =
TRUE), .groups = ``drop'') \%\textgreater\% group\_by(COD\_BARCO, year)
\%\textgreater\% mutate(species\_share = total\_catch /
sum(total\_catch, na.rm = TRUE)) \%\textgreater\% ungroup()
\%\textgreater\% select(-total\_catch) \%\textgreater\%
complete(COD\_BARCO, year, NOMBRE\_ESPECIE, fill = list(species\_share =
0)) \%\textgreater\% group\_by(COD\_BARCO, year) \%\textgreater\%
mutate(species\_share\_check = sum(species\_share, na.rm = TRUE))
\%\textgreater\% filter(species\_share\_check \textgreater{} 0)
\%\textgreater\% dplyr::select(-c(species\_share\_check))
\%\textgreater\% group\_by(COD\_BARCO, NOMBRE\_ESPECIE) \%\textgreater\%
summarize(species\_share = mean(species\_share, na.rm = TRUE))
\%\textgreater\% pivot\_wider( names\_from = NOMBRE\_ESPECIE,
values\_from = species\_share, values\_fill = 0) \%\textgreater\%
dplyr::select(c(COD\_BARCO, ANCHOVETA, \texttt{SARDINA\ COMUN}, JUREL))
\%\textgreater\% mutate(OTHER = 1 - ANCHOVETA - \texttt{SARDINA\ COMUN}
- JUREL) \%\textgreater\% ACA DEBERIA CALCULAR TODAS LAS DEMAS
ESPECIES!!! rename( Anchoveta = \texttt{ANCHOVETA}, Sardine =
\texttt{SARDINA\ COMUN}, JackMackerel = \texttt{JUREL}, Other =
\texttt{OTHER} )

get\_strategy \textless- function(sardine, jackmackerel, anchoveta,
other) \{ species \textless- c() if (sardine \textgreater{} 0.20)
species \textless- c(species, ``Sardine'') if (jackmackerel
\textgreater{} 0.20) species \textless- c(species, ``JackMackerel'') if
(anchoveta \textgreater{} 0.20) species \textless- c(species,
``Anchoveta'') if (other \textgreater{} 0.20) species \textless-
c(species, ``Other'')

n \textless- length(species) if (n == 0) return(``None or negligible'')
if (n == 1) return(paste(``Only'', species{[}1{]})) if (n == 2)
return(paste(species{[}1{]}, ``and'', species{[}2{]})) if (n == 3)
return(paste0(species{[}1{]}, ``,'', species{[}2{]}, '' and ``,
species{[}3{]})) return(paste0(species{[}1{]},'', ``, species{[}2{]},'',
``, species{[}3{]},'' and ``, species{[}4{]})) \}

vessel\_species\_year \textless- vessel\_species\_year \%\textgreater\%
mutate(strategy\_After = pmap\_chr( list(Sardine, JackMackerel,
Anchoveta, Other), get\_strategy ))

strategy\_percent \textless- vessel\_species\_year \%\textgreater\%
group\_by(strategy\_After) \%\textgreater\% summarise(n = n(), .groups =
``drop'') \%\textgreater\% mutate(percent = round(100 * n / sum(n), 1))
\%\textgreater\% arrange(desc(percent))

strategy\_after \textless- vessel\_species\_year \%\textgreater\%
select(COD\_BARCO, strategy\_After)

\begin{verbatim}
```{r harvest_substitution_before, message=FALSE, warning=FALSE, include=FALSE}

# First, get your original summarized table
vessel_species_year <- logbooks %>% 
  filter(year >= 2012, year < 2019) %>% 
  select(COD_BARCO, year, NOMBRE_ESPECIE, CAPTURA_RETENIDA) %>%
  group_by(COD_BARCO, year, NOMBRE_ESPECIE) %>%
  summarize(total_catch = sum(CAPTURA_RETENIDA, na.rm = TRUE), .groups = "drop") %>%
  group_by(COD_BARCO, year) %>%
  mutate(species_share = total_catch / sum(total_catch, na.rm = TRUE)) %>%
  ungroup() %>%
  select(-total_catch) %>%
  complete(COD_BARCO, year, NOMBRE_ESPECIE, fill = list(species_share = 0)) %>%
  group_by(COD_BARCO, year) %>%
  mutate(species_share_check = sum(species_share, na.rm = TRUE)) %>%
  filter(species_share_check > 0) %>%
  dplyr::select(-c(species_share_check)) %>%
  group_by(COD_BARCO, NOMBRE_ESPECIE) %>% 
  summarize(species_share = mean(species_share, na.rm = TRUE)) %>%
  pivot_wider(
    names_from = NOMBRE_ESPECIE,
    values_from = species_share,
    values_fill = 0) %>%
  dplyr::select(c(COD_BARCO, ANCHOVETA, `SARDINA COMUN`, JUREL)) %>%
  mutate(OTHER = 1 - ANCHOVETA - `SARDINA COMUN` - JUREL) %>%
  rename(
    Anchoveta = `ANCHOVETA`,
    Sardine = `SARDINA COMUN`,
    JackMackerel = `JUREL`,
    Other = `OTHER`
  )

vessel_species_year_pre <- vessel_species_year %>%
  mutate(strategy_Before = pmap_chr(
    list(Sardine, JackMackerel, Anchoveta, Other),
    get_strategy
  ))

strategy_percent_pre <- vessel_species_year_pre %>%
  group_by(strategy_Before) %>%
  summarise(n = n(), .groups = "drop") %>%
  mutate(percent = round(100 * n / sum(n), 1)) %>%
  arrange(desc(percent)) 

strategy_before <- vessel_species_year_pre %>%
  select(COD_BARCO, strategy_Before)
\end{verbatim}

See Figure @ref(fig:switchStrategy) and Table
@ref(tab:strategy-table-logbooks) for strategy transitions.

```\{r harvest\_SERNAPESCA, message=FALSE, warning=FALSE,
include=FALSE\}

library(readxl) library(dplyr)

annual\_harvest\_SERNAPESCA\_ART \textless- read\_excel(paste0(dirdata,
``SERNAPESCA/AH010T0006857\_sobre\_desembarque\_pelagicos\_2012\_2024.xlsx''),
sheet = ``ART\_2012\_2024'', range = ``A6:S36337'') \%\textgreater\%
rename(specie = Especie, year = Año, region =
\texttt{Región\ de\ Operación}) \%\textgreater\% mutate(zone =
case\_when( region \%in\% c(1,2,3,4,15) \textasciitilde{} ``Norte'',
region \%in\% c(5,6,7,8,9,10,14,16) \textasciitilde{} ``Centro\_Sur'',
region \%in\% c(11,12) \textasciitilde{} ``Extremo\_Sur'', TRUE
\textasciitilde{} ``No\_Especifica'')) \%\textgreater\%
group\_by(specie, year, zone) \%\textgreater\%
summarize(annual\_harvest\_ART\_SERNAPESCA = sum(SumaDeDesembarque,
na.rm = TRUE), .groups = ``drop'')

annual\_harvest\_SERNAPESCA\_IND \textless- read\_excel(paste0(dirdata,
``SERNAPESCA/AH010T0006857\_sobre\_desembarque\_pelagicos\_2012\_2024.xlsx''),
sheet = ``IND\_2012\_2024'', range = ``A6:R3349'') \%\textgreater\%
rename(specie = Especie, year = Año, region = \texttt{Región})
\%\textgreater\% mutate(zone = case\_when( region \%in\% c(1,2,3,4,15)
\textasciitilde{} ``Norte'', region \%in\% c(5,6,7,8,9,10,14,16)
\textasciitilde{} ``Centro\_Sur'', region \%in\% c(11,12)
\textasciitilde{} ``Extremo\_Sur'', TRUE \textasciitilde{}
``No\_Especifica'' )) \%\textgreater\% group\_by(specie, year, zone)
\%\textgreater\% summarize(annual\_harvest\_IND\_SERNAPESCA =
sum(Desembarque, na.rm = TRUE), .groups = ``drop'')

annual\_harvest\_SERNAPESCA\_BF \textless- read\_excel(paste0(dirdata,
``SERNAPESCA/AH010T0006857\_sobre\_desembarque\_pelagicos\_2012\_2024.xlsx''),
sheet = ``BF\_2017\_2024'', range = ``A7:R144'') \%\textgreater\%
rename(specie = DESCR1TABL, year = Año, region = \texttt{Cd\_Region})
\%\textgreater\% mutate(zone = case\_when( region \%in\% c(1,2,3,4,15)
\textasciitilde{} ``Norte'', region \%in\% c(5,6,7,8,9,10,14,16)
\textasciitilde{} ``Centro\_Sur'', region \%in\% c(11,12)
\textasciitilde{} ``Extremo\_Sur'', TRUE \textasciitilde{}
``No\_Especifica'' )) \%\textgreater\% group\_by(specie, year, zone)
\%\textgreater\% summarize(annual\_harvest\_BF\_SERNAPESCA =
sum(DESEMBARQUE, na.rm = TRUE), .groups = ``drop'')

harvest\_SERNAPESCA \textless-
left\_join(annual\_harvest\_SERNAPESCA\_ART,
annual\_harvest\_SERNAPESCA\_IND, by = c(``year'', ``specie'',
``zone'')) \%\textgreater\% left\_join(annual\_harvest\_SERNAPESCA\_BF,
by = c(``year'', ``specie'', ``zone'')) \%\textgreater\%
mutate(total\_harvest\_SERNAPESCA =
rowSums(across(c(annual\_harvest\_IND\_SERNAPESCA,
annual\_harvest\_ART\_SERNAPESCA)), na.rm = TRUE)) \%\textgreater\%
mutate(total\_harvest\_all\_SERNAPESCA =
rowSums(across(c(annual\_harvest\_IND\_SERNAPESCA,
annual\_harvest\_BF\_SERNAPESCA, annual\_harvest\_ART\_SERNAPESCA)),
na.rm = TRUE))

harvest\_SERNAPESCA \textless- harvest\_SERNAPESCA \%\textgreater\%
pivot\_wider( names\_from = zone, values\_from = c(
annual\_harvest\_ART\_SERNAPESCA, annual\_harvest\_IND\_SERNAPESCA,
annual\_harvest\_BF\_SERNAPESCA, total\_harvest\_SERNAPESCA,
total\_harvest\_all\_SERNAPESCA ), names\_sep = ``\_'' )

rm(list = c(``annual\_harvest\_SERNAPESCA\_ART'',
``annual\_harvest\_SERNAPESCA\_IND'',
``annual\_harvest\_SERNAPESCA\_BF''))

\begin{verbatim}
```{r harvest_IFOP, message=FALSE, warning=FALSE, include=FALSE}

library(readxl)
library(dplyr)
library(ggplot2)
library(tidyr)

### Industriales 

harvest_IFOP_jmck_IND_month <- 
  read_excel(paste0(dirdata, "IFOP/4. DESEMBARQUES.xlsx"), 
             sheet = "INDUSTRIAL (nacional)", 
             range = "A2:L293") %>% 
  mutate(specie = "JUREL") %>%
  rename(year = `Años (Fc_Llegada)`) %>%
  rename(month = `Meses (Fc_Llegada)`) %>%
  mutate(harvest_IND_IFOP_norte = rowSums(across(c('15', '1', '2', '3', '4')), na.rm = TRUE)) %>%
  mutate(harvest_IND_IFOP_centrosur = rowSums(across(c('5', '8', '14', '10')), na.rm = TRUE)) %>% 
  dplyr::select(year, month, specie, harvest_IND_IFOP_norte, harvest_IND_IFOP_centrosur)

harvest_IFOP_sardine_IND_month <- 
  read_excel(paste0(dirdata, "IFOP/4. DESEMBARQUES.xlsx"), 
             sheet = "INDUSTRIAL (nacional)", 
             range = "N2:W196") %>% 
  mutate(specie = "SARDINA COMUN") %>%
  rename(year = `Años (Fc_Llegada)`) %>%
  rename(month = `Meses (Fc_Llegada)`) %>%
  mutate(harvest_IND_IFOP_norte = rowSums(across(c('1', '2', '4')), na.rm = TRUE)) %>%
  mutate(harvest_IND_IFOP_centrosur = rowSums(across(c('5', '8', '14', '10')), na.rm = TRUE)) %>% 
  dplyr::select(year, month, specie, harvest_IND_IFOP_norte, harvest_IND_IFOP_centrosur)

harvest_IFOP_anchoveta_IND_month <- 
  read_excel(paste0(dirdata, "IFOP/4. DESEMBARQUES.xlsx"), 
             sheet = "INDUSTRIAL (nacional)", 
             range = "Y2:AJ273") %>% 
  mutate(specie = "ANCHOVETA") %>%
  rename(year = `Años (Fc_Llegada)`) %>%
  rename(month = `Meses (Fc_Llegada)`) %>%
  mutate(harvest_IND_IFOP_norte = rowSums(across(c('15', '1', '2', '3', '4')), na.rm = TRUE)) %>%
  mutate(harvest_IND_IFOP_centrosur = rowSums(across(c('5', '8', '14', '10')), na.rm = TRUE)) %>% 
  dplyr::select(year, month, specie, harvest_IND_IFOP_norte, harvest_IND_IFOP_centrosur)

harvest_IFOP_IND_month <- rbind(harvest_IFOP_jmck_IND_month, harvest_IFOP_sardine_IND_month, harvest_IFOP_anchoveta_IND_month)
rm(list = c("harvest_IFOP_jmck_IND_month", "harvest_IFOP_sardine_IND_month", "harvest_IFOP_anchoveta_IND_month"))


### Lanchas

harvest_IFOP_jmck_LANCHAS_month <- 
  read_excel(paste0(dirdata, "IFOP/4. DESEMBARQUES.xlsx"), 
             sheet = "LANCHAS (CentroSur)", 
             range = "A2:I143") %>% 
  mutate(specie = "JUREL") %>%
  rename(year = `Años (Fc_Llegada)`) %>%
  rename(month = `Meses (Fc_Llegada)`) %>%
  mutate(harvest_LANCHAS_IFOP_centrosur = rowSums(across(c('5', '7', '8', '9', '14', '10')), na.rm = TRUE)) %>% 
  dplyr::select(year, month, specie, harvest_LANCHAS_IFOP_centrosur)

harvest_IFOP_sardine_LANCHAS_month <- 
  read_excel(paste0(dirdata, "IFOP/4. DESEMBARQUES.xlsx"), 
             sheet = "LANCHAS (CentroSur)", 
             range = "K2:S170") %>% 
  mutate(specie = "SARDINA COMUN") %>%
  rename(year = `Años (Fc_Llegada)`) %>%
  rename(month = `Meses (Fc_Llegada)`) %>%
  mutate(harvest_LANCHAS_IFOP_centrosur = rowSums(across(c('5', '7', '8', '9', '14', '10')), na.rm = TRUE)) %>% 
  dplyr::select(year, month, specie, harvest_LANCHAS_IFOP_centrosur)

harvest_IFOP_anchoveta_LANCHAS_month <- 
  read_excel(paste0(dirdata, "IFOP/4. DESEMBARQUES.xlsx"),  
             sheet = "LANCHAS (CentroSur)", 
             range = "U2:AC169") %>% 
  mutate(specie = "ANCHOVETA") %>%
  rename(year = `Años (Fc_Llegada)`) %>%
  rename(month = `Meses (Fc_Llegada)`) %>%
  mutate(harvest_LANCHAS_IFOP_centrosur = rowSums(across(c('5', '8', '9', '14', '10')), na.rm = TRUE)) %>% 
  dplyr::select(year, month, specie, harvest_LANCHAS_IFOP_centrosur)

harvest_IFOP_LANCHAS_month <- rbind(harvest_IFOP_jmck_LANCHAS_month, 
                              harvest_IFOP_sardine_LANCHAS_month, 
                              harvest_IFOP_anchoveta_LANCHAS_month)
rm(list = c("harvest_IFOP_jmck_LANCHAS_month", 
            "harvest_IFOP_sardine_LANCHAS_month", 
            "harvest_IFOP_anchoveta_LANCHAS_month"))

### Botes

harvest_IFOP_jmck_BOTES_month <- 
  read_excel(paste0(dirdata, "IFOP/4. DESEMBARQUES.xlsx"), 
             sheet = "BOTES (CentroSur)", 
             range = "A2:I146") %>% 
  mutate(specie = "JUREL") %>%
  rename(year = `Años (Fc_Llegada)`) %>%
  rename(month = `Meses (Fc_Llegada)`) %>%
  mutate(harvest_BOTES_IFOP_centrosur = 
           rowSums(across(c('5', '7', '8', '14', '10')), na.rm = TRUE)) %>% 
  dplyr::select(year, month, specie, harvest_BOTES_IFOP_centrosur)

harvest_IFOP_sardine_BOTES_month <- 
  read_excel(paste0(dirdata, "IFOP/4. DESEMBARQUES.xlsx"), 
             sheet = "BOTES (CentroSur)", 
             range = "K2:S163") %>% 
  mutate(specie = "SARDINA COMUN")  %>%
  rename(year = `Años (Fc_Llegada)`) %>%
  rename(month = `Meses (Fc_Llegada)`) %>%
  mutate(harvest_BOTES_IFOP_centrosur = 
           rowSums(across(c('5', '7', '8', '9', '14', '10')), na.rm = TRUE)) %>% 
  dplyr::select(year, month, specie, harvest_BOTES_IFOP_centrosur)

harvest_IFOP_anchoveta_BOTES_month <- 
  read_excel(paste0(dirdata, "IFOP/4. DESEMBARQUES.xlsx"), 
             sheet = "BOTES (CentroSur)", 
             range = "U2:AD109") %>% 
  mutate(specie = "ANCHOVETA") %>%
  rename(year = `Años (Fc_Llegada)`) %>%
  rename(month = `Meses (Fc_Llegada)`) %>%
  mutate(harvest_BOTES_IFOP_centrosur = 
           rowSums(across(c('5', '7', '8', '9', '14', '10')), na.rm = TRUE)) %>% 
  dplyr::select(year, month, specie, harvest_BOTES_IFOP_centrosur) 

harvest_IFOP_BOTES_month <- 
  rbind(harvest_IFOP_jmck_BOTES_month,
        harvest_IFOP_sardine_BOTES_month, 
        harvest_IFOP_anchoveta_BOTES_month)
rm(list = c("harvest_IFOP_jmck_BOTES_month", 
            "harvest_IFOP_sardine_BOTES_month", 
            "harvest_IFOP_anchoveta_BOTES_month"))


### Unir Industriales, botes y lanchas
harvest_IFOP_month <- 
  full_join(harvest_IFOP_LANCHAS_month, harvest_IFOP_BOTES_month, by = c("year", "specie", "month")) %>%
  full_join(harvest_IFOP_IND_month, by = c("year", "specie", "month")) %>%
  mutate(total_harvest_IFOP_centrosur = 
           rowSums(
             across(
               c(harvest_IND_IFOP_centrosur,
                 harvest_LANCHAS_IFOP_centrosur, 
                 harvest_BOTES_IFOP_centrosur)), 
             na.rm = TRUE)) %>%
  mutate(total_harvest_IFOP_norte = harvest_IND_IFOP_norte) %>% filter(year >= 2012)

harvest_IFOP <- harvest_IFOP_month %>%
  group_by(specie, year) %>%
  summarise(
    harvest_LANCHAS_IFOP_centrosur = sum(harvest_LANCHAS_IFOP_centrosur, na.rm = TRUE),
    harvest_BOTES_IFOP_centrosur = sum(harvest_BOTES_IFOP_centrosur, na.rm = TRUE),
    harvest_IND_IFOP_norte = sum(harvest_IND_IFOP_norte, na.rm = TRUE),
    harvest_IND_IFOP_centrosur = sum(harvest_IND_IFOP_centrosur, na.rm = TRUE),
    total_harvest_IFOP_centrosur = sum(total_harvest_IFOP_centrosur, na.rm = TRUE),
    total_harvest_IFOP_norte = sum(total_harvest_IFOP_norte, na.rm = TRUE),
    .groups = "drop"
  )

rm(list = c("harvest_IFOP_IND_month", "harvest_IFOP_BOTES_month", "harvest_IFOP_LANCHAS_month"))
\end{verbatim}

\hypertarget{data-and-methodology}{%
\section{Data and methodology}\label{data-and-methodology}}

To fulfill the project's objectives, and following @Kasperski2015-jm,
the research entails five different stages: (i) estimating the stock
dynamics of each species included in the model, (ii) estimating trip
level cost functions, (iii) estimating total annual trips, (iv) estimate
the inverse demand model for outputs (i.e., price responses to supply),
and (v) conduct numerical optimization to examine how harvest and
profits levels evolve over time. The numeral optimization uses estimated
parameters from the previous four stages to conduct the optimization
procedure.

\hypertarget{data}{%
\subsection{Data}\label{data}}

\begin{itemize}
\tightlist
\item
  \textbf{SOLICITADO A IFOP 2012-2024:}

  \begin{itemize}
  \tightlist
  \item
    Stock abundance and vessel landings (annual by
    port/county/region/country and species)
  \item
    Data at the trip level (\href{ifop.dataobservatory.net}{IFOP data
    observatory?}{]}).
  \item
    Ex-vessel prices (monthly or annual by port/county/region/country
    and species)
  \end{itemize}
\end{itemize}

How different are SERNAPESCA and IFOP harvest data? (Figura
@ref(fig:harvestsource))

\begin{itemize}
\tightlist
\item
  \textbf{POR SOLICITAR:}

  \begin{itemize}
  \tightlist
  \item
    Environmental covariates -- Ask Fabian Tapia, UdeC

    \begin{itemize}
    \tightlist
    \item
      Sea surface temperature
    \item
      Chlorophyll levels
    \item
      Wind intensity and wave conditions in each trip at the harvest
      location
    \item
      Bad weather days?
    \end{itemize}
  \item
    Other data?

    \begin{itemize}
    \tightlist
    \item
      Average wage pay to crew member per hour
    \item
      Diesel cost.
    \item
      Permits by vessels
    \item
      Quota prices?
    \end{itemize}
  \end{itemize}
\item
  @Birkenbach2024:

  \begin{itemize}
  \tightlist
  \item
    \emph{Day at sea price captures elements of forward-looking behavior
    and information. @reimer2022structural similarly argue that
    including a quota price captures forward looking behavior and allows
    one to simplify the dynamic model to a static one.}
  \item
    \emph{Data on wind speed and direction were collected from NOAA's
    National Center for Atmospheric Prediction's high-resolution North
    American Regional Reanalysis dataset and averaged to the daily level
    for each stock centroid location, defined as gear and month-specific
    average latitudes and longitudes where fishing occurs for each
    stock.}
  \end{itemize}
\end{itemize}

\hypertarget{econometrics-models}{%
\subsection{Econometrics models}\label{econometrics-models}}

\hypertarget{stock-dynamics}{%
\subsubsection{Stock dynamics}\label{stock-dynamics}}

To estimate stock dynamics, I use annual data on stock abundance and
vessel landings. Following @Kasperski2015-jm, the growth of each species
follows a discrete logistic function: \begin{equation}
x_{i,y+1} + h_{iy} = \underbrace{(1 + r_i)x_{iy} + \eta_i x_{iy}^2}_{R_i(x_{iy})} + \underbrace{\sum_{j \neq i}^{n-1} a_{ij} x_{iy} x_{jy}}_{I_i(x_y)} \quad i=1,\ldots,n
\label{eq1}
\end{equation}

where \(x_{iy}\) is the fish stock by species \(i=1,\ldots,n\) in year
\(y\), \(n\) is the total number of species, \(h_{iy}\) is the annual
harvest of species \(i\) on year \(y\), \(r_i\) is the intrinsic growth
rate of the resource \(i\), \(\eta_i\) is a density-dependent factor
related to the carrying capacity, and \(\alpha_{ij}\) are the
interaction parameters between species. The system of \(n\) growth
equations can be estimated simultaneously using seemingly unrelated
regression (SUR) or other similar approaches.

Following @Richter2018, we can augment \eqref{eq1} by including
environmental covariates \(Env_{iy}\) that affect the fish stock, such
as sea surface temperature and chlorophyll levels, and an error term
\(\varepsilon_{iy}\) that captures random recruitment: \begin{equation}
x_{i,y+1} + h_{iy} = \underbrace{(1 + r_i)x_{iy} + \eta_i x_{iy}^2}_{R_i(x_{iy})} + \underbrace{\sum_{j \neq i}^{n-1} a_{ij} x_{iy} x_{jy}}_{I_i(x_y)} + \rho_i Env_{iy} + \varepsilon_{iy} \quad i=1,\ldots,n
\label{eq2}
\end{equation} where \(\rho_i\) are the coefficient for the
environmental covariates. The model could also be expanded to different
spatial locations conditional on data availability.

As shown in Figure @ref(fig:biomass), biomass levels vary by species,
and there is some interrelation between them. It is also clear that
these biomass levels are affected by the harvests that occurred during
those periods. For instance, in the case of jack mackerel, an abrupt
decline in biomass is observed, likely due to a combination of
overexploitation of the resource and unfavorable environmental
conditions.

\begin{itemize}
\tightlist
\item
  From {[}@Yáñez2014{]}:

  \begin{itemize}
  \tightlist
  \item
    \emph{``@Fuenzalida2007 forecast that surface winds would strengthen
    in the coast of Chile, with an increase of 6 m/s in some areas of
    Chile during the period 2046-2065 in comparison to 2000--2005; this
    might increase upwelling and thus, fisheries productivity
    {[}@garreaud2009{]}.''}

    \begin{itemize}
    \tightlist
    \item
      \emph{``Wind direction and strength will probably influence the
      distribution and abundance of marine species. Small and coastal
      pelagic species, for example, show different behaviors: while
      anchovy maximizes recruitment at current speeds of 5.46 m/s,
      showing an important decrease with lower and higher values,
      sardine maximizes recruitment at 5.63 m/s or more
      {[}@yanez2001{]}.}
    \end{itemize}
  \item
    \emph{Anchovy dominates during cold inter-decadal periods, while
    sardine prevail during warm inter-decadal periods. Such interdecadal
    variations in SST also influence recruitment, a situation that has
    been documented in anchovies off the Peruvian coast
    {[}@cahuin2009{]}.''}

    \begin{itemize}
    \tightlist
    \item
      \emph{Longer term predictions based on two global warming
      scenarios of the IPCC (Intergovernmental Panel on Climate Change)
      done by @Fuenzalida2007 shows a warming on the Chilean coast.''}
    \end{itemize}
  \end{itemize}
\end{itemize}

Adding harvest:

\hypertarget{trip-level-cost-functions}{%
\subsubsection{Trip level cost
functions}\label{trip-level-cost-functions}}

Ignoring trip subscript, the cost functions vary by vessel
\(v=1,\ldots,V_g\) and gear used \(g=1,\ldots,G\), where \(V_g\) is the
number of observations using gear type \(g\), and \(G\) is the total
number of available (or observed) gears: \begin{equation}
C_{vg} = \sum_{i=1}^{2n+M+k} \alpha_{g, \mathbf{X}_i} \mathbf{X}_{ivg} + \frac{1}{2} \sum_{i=1}^{2n+M+k} \sum_{j=1}^{2n+M+k} \alpha_{g, \mathbf{X}_i\mathbf{X}_j} \mathbf{X}_{ivg} \mathbf{X}_{jvg}
\label{eq3}
\end{equation} where \(C_{vg}= w z_{vg}^*\),
\(\mathbf{X}_{vg}=[w;h_{vg};x;Z_v]\), \(w\) is a \(V_g \times M\) matrix
of variable input prices, \(h_{vg}\) is an \(V_g \times n\) matrix of
harvest quantities, \(x\) is an \(V_g \times n\) matrix of given stock
levels of the species of interest, and \(Z_v\) is an \(V_g \times k\)
matrix of given vessel characteristics. Therefore, \(\mathbf{X}_{vg}\)
is a \(V_g \times (2n+M+k)\) matrix, and \(\mathbf{X}_{ivg}\) represents
the \emph{i}th column of the \(\mathbf{X}_{vg}\) matrix.

Together with estimating the restricted cost function, we estimate the
conditional input demand equations. This addition allows an increase in
the degrees of freedom by imposing cross-equation parameter constraints
and allows for the testing of, for instance, jointness in inputs
{[}@Kasperski2015-jm{]}. The conditional input demand equations are
derived by Shepard's Lemma: \begin{equation}
\frac{\partial C_{vg}}{\partial w_m} = z^*_{vg,w_m} = \alpha_{g,w_m} + \sum_{j=1}^{2n+M+k} \alpha_{g,w_m,\mathbf{X}_j} \mathbf{X}_{jvg} \quad m=1,\ldots,M.
\label{eq4}
\end{equation}

Similar to stock dynamics, the system of equations formed by \eqref{eq3}
and \eqref{eq4} can be estimated using SUR. To comply with economic
theory, and to reduce even more the number of parameters to estimate,
the following restrictions are imposed when estimating \eqref{eq4}:

\begin{enumerate}
\def\labelenumi{\arabic{enumi}.}
\item
  Symmetry of the cost function, where \begin{equation*}
    \alpha_{g,\mathbf{X}_i\mathbf{X}_j} = \alpha_{g,\mathbf{X}_j\mathbf{X}_i} \quad \forall \quad i=1,\ldots,(2n + M + k); \ i \neq j; \ g = 1,\ldots, G.
    \end{equation*}
\item
  Linear homogeneity in input prices, where \begin{equation*}
    \sum_m^M \alpha_{g,w_m} = 1 \ \text{and} \ \sum_m^M \alpha_{g,w_m,\mathbf{X}_i} = 0 \quad i=1,\ldots,(2n + M + k); \ g = 1,\ldots, G.
    \end{equation*}
\end{enumerate}

Data at the trip level is available upon request from the Chilean
Fisheries Research Institute (IFOP), which registers geo-referenced
catch information on the Chilean fleet's fishing operation per trip
{[}see e.g. @Pena-Torres2017-gn and {[}IFOP data
observatory{]}(ifop.dataobservatory.net){]}. As inputs we can use the
time spent at sea during a trip, where the price is the average wage pay
to crew member per hour, and the distance traveled during a trip, where
the price of distance traveled is the diesel cost. Therefore, the total
cost function \(C_{vg}= w z_{vg}^*\) for vessel \(v\), using gear \(g\)
in a trip would be sum of the total cost of distance travelled plus the
total cost of the time spent at sea.

\textbf{Note:} \emph{Depending on the type of vessel, this cost should
change. Some vessels are more efficient, other one are more heavy. How
to capture this? The righ hand side has vessel characteristics, so the
effect of harvest would be conditional on vessel characteristics, the
stock levels and input prices. As we only care in the margin how harvest
increase cost, this should be fine. @Kasperski2015-jm mention this
``\ldots no reliable fixed cost information on these vessels exists, but
these should not affect the optimization as economic decisions are made
at the margin. Therefore, this study does not measure true profit, but
rather a proxy based on the net operating rent accruing to vessels in
the fishery.''}

To link this function to climate change, we can also include additional
environmental variables \(Env\) to \(\mathbf{X}_{vg}\) such as wind
intensity and wave conditions in each trip at the harvest location, upon
data availability. Therefore, the augmented \(X_{vg}\) matrix becomes
\(\mathbf{X}^{'}_{vg} = [w;h_{vg};x;Z_v;Env]\).

MAYBE INCLUDE QUOTA PRICE? \textbf{Higher quota prices for depleted
stocks (e.g., GOM cod) reduce incentives to target them.} and
\textbf{Active leasing markets for quota (and previously for DAS) allow
fishermen to treat quota as a ``priced input'' rather than a fixed,
exhaustible resource.}

\hypertarget{total-annual-trips}{%
\subsubsection{Total annual trips}\label{total-annual-trips}}

The number of trips a vessel will take in a given year for each gear
type used is assumed to follow a Poison distribution
{[}@Kasperski2015-jm{]}: \begin{equation}
Pr\left[T^{*}_{vgy} = t_v\right] = \frac{exp^{-exp(U^{'}_{vg}\beta_g)}exp(U^{'}_{vg}\beta_g)^{t_v}}{t_v !}
\label{eq5}
\end{equation} where \(U_{vg}=[p,w,h_{vg},\bar{q},Z_{vg}]\) is a
\((3n+M+k+1)×V_g\) matrix of explanatory variables, \(\beta_g\) is a
\((3n+M+k+1)\times1\) matrix of coefficients to be estimated, \(t_v\) is
the number of trips taken by vessel \(v\) using gear type \(g\) in year
\(y\), and \(\bar{q}\) is the annual quota level. Additionally, we can
add the accumulation of ``bad weather days'' as an explanatory variable
to incorporate weather conditions into this decision, thus
\(U_{vg}=[p,w,h_{vg},\bar{q},Z_{vg}, Env]\)

\hypertarget{inverse-demand-model-for-outputs}{%
\subsubsection{Inverse demand model for
outputs}\label{inverse-demand-model-for-outputs}}

The price of each species is modeled using an inverse demand model,
which assumes weak separability between the species into consideration
and other products {[}@Kasperski2015-jm{]}. The price of a species \(i\)
in year \(y\) is the following:

\begin{equation}
p_{iy} = \sum_j^n \gamma_j p_{j,y-1} + \gamma_{h_i} h_{iy} + \epsilon_{iy}, \quad i = 1,\ldots,n, \ j = 1,\ldots,n.
\label{eq6}
\end{equation}

The system formed by \eqref{eq6} can be estimated using maximum
likelihood. Note that harvest may be endogenous in this system due to
simultaneity. @Kasperski2015-jm solves this by assuming that the TAC is
exogenous, and the catch, in general, is determined by this quota. We
can relax this assumption by considering that all variables in the
inverse demand equations are endogenous by estimating a vector
autoregressive (VAR) model {[}@juselius2006cointegrated{]}. In other
words, harvest \(h_{vg}\) has its own equations in the system.

\hypertarget{numerical-optimization}{%
\subsection{Numerical optimization}\label{numerical-optimization}}

To obtain the effect of future climate variability on stock, harvest,
quota and profits, we conduct numerical optimization for different
climate scenarios using the parameters estimates for the stock dynamic,
cost functions, total annual trips and inverse demand equations. In each
year, a vessel maximizes profits by choosing their optimal number of
trips \(T_g\) and harvest levels per trip \(h_{g\tau}\) given a gear
type: \begin{align}
\max_{h_{gt}, T_g} \quad \pi_{vgt} & = \sum_{\tau=t}^{T_g} \rho^\tau \left\{ P(h) h_{g\tau} - C_g(h_{g\tau} | w, x, Z, Env) \right\} \quad \tau = t,\ldots, T_g \nonumber \\
\textbf{s.t} \quad q_{g,t+1} & = \omega \ast \bar{q} - \sum_{t=1}^{t} h_{gt} \geq 0, \quad t = 1, \dots, T-1, \quad g = 1, \dots, G
\label{eq7}
\end{align} where \(\rho\) is the intra-annual discount factor,
\(\omega\) is a vector of shares of \(\bar{q}\), and \(h_{lt}=0\) for
all \(l\neq g\). The vector of shares is obtained from historical data
on harvest. The optimal profit from the maximization problem in
\eqref{eq7} is denoted as \(\pi_{vgy}^* (p,w,x,Z,\bar{q},\omega, Env)\),
and \(h_{vgty}^*\) and \(T_{vgy}^*\) are the optimal choices harvest per
trip and total number of trips in year \(y\) for vessel \(v\). To obtain
the optimal quota level, we must solve the social-planner optimization
problem to maximize the net value of the fishery by choosing the quota
levels per year and by species.

Following @Kasperski2015-jm, the optimization problem will be conducted
for the next 25 years. I will use different climate scenarios and
compare different optimal outcomes between them by using future
projections for the environmental variables included in the model.

\hypertarget{projections}{%
\subsection{Projections}\label{projections}}

\emph{From {[}@Yáñez2014{]}}:

\begin{itemize}
\tightlist
\item
  ``To project the model, the average structure of catches and
  temperature (of Antofagasta and the region Niño 3 + 4) for the years
  2005, 2006 and 2007 were used as starting point. We consider a linear
  increase in temperature, taking into account four climate change
  scenarios based on the scenarios presented by IPCC, designed for the
  northern part of Chile until 2100.

  \begin{enumerate}
  \def\labelenumi{\arabic{enumi}.}
  \tightlist
  \item
    The first scenario considers an increase in temperature of 0.034ºC
    per year {[}@Fuenzalida2007{]}, similar to that estimated by

    \begin{enumerate}
    \def\labelenumii{\arabic{enumii}.}
    \tightlist
    \item
    \end{enumerate}
  \item
    A second scenario, more moderate, of 0,025ºC/year is also proposed
    by @Fuenzalida2007.
  \item
    The third scenario is not considered a significant effect on the
    area, following the work of 1.
  \item
    The fourth scenario is contradictory, indicating a cooling de 0.02
    ºC/year {[}@falvey2009{]}. It should be noted that according to the
    work of @Fuenzalida2007 and @falvey2009 the same SST increase (or
    decrease) were considered for both temperatures (in Antofagasta and
    in the Niño 3 + 4 region).''
  \end{enumerate}
\end{itemize}

\hypertarget{results}{%
\section{Results}\label{results}}

NO RESULTS YET

\hypertarget{discussion}{%
\section{Discussion}\label{discussion}}

If ITQ in this fishery in Chile: \textbf{The theoretical findings on
multispecies harvest patterns in @Birkenbach2020-nh give rise to nuanced
hypotheses about how behavior and outcomes will change after the
adoption of catch shares. For example, a secure property right to catch
fish at any time in the fishing season allows firms to spread the catch
of stocks with high prices and downward-sloping demand over a longer
fishing season. This minimizes market gluts that steer product toward
lower priced frozen markets {[}@homans2005{]} and can result in higher
prices for those species. Fishermen might also shift their efforts
toward lower-priced species with cheaper quota or toward non-catch-share
fisheries, intensifying the race to fish for those species during
portions of the season {[}@asche2007; @cunningham2016spillovers{]}.} --
However, we do not include other species than jack mackerel, sardine and
anchovy that might be caught by thise fleet. This would require to
expand the model by \textbf{N} species, which would increase
dimensionality of the model. WE NEED PERMITS TO CHECK IF ACTUALLY THIS
HAPPEN! (Stil problem with Open-Access)

\hypertarget{potential-extension-of-the-project}{%
\subsection{Potential extension of the
project}\label{potential-extension-of-the-project}}

Several other extensions to the model can be incorporated to be
improved. For instance, the geographical space where fishermen operate
is relevant, as depending on the location chosen and when to
participate, the set of potential choices would vary
{[}@Reimer2017-jw{]}. As I mentioned above, it is possible to extend the
stock dynamic model by considering different locations. The model would
also require that the participation decision, which is captured by the
Poisson model on the annual number of trips, should then consider the
decision to participate in a determined fishing ground, connecting the
multi-species model of @Kasperski2015-jm to the literature of location
choice modeling {[}e.g., @Dupont1993-jn; @Smith2005-us;
@Hicks2020-mz{]}.

\hypertarget{damage-function-for-the-fisheries-sector}{%
\subsection{Damage function for the fisheries
sector}\label{damage-function-for-the-fisheries-sector}}

Link to the work made in the U.S. West Coast. Similar weather, but
different development. We would need to also have estimate of the
dose-response function in other latitudes, with significantly different
temperatures\ldots{}

\hypertarget{conclusions}{%
\section{Conclusions}\label{conclusions}}

NO CONCLUSION YET

\hypertarget{repository}{%
\section{Repository}\label{repository}}

The source code for this project is available on
\href{https://github.com/fquezadae/Impact-of-Environmental-Variability-on-Harvest}{GitHub}

\hypertarget{references}{%
\section{References}\label{references}}
